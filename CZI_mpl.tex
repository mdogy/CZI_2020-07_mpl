\documentclass[12pt,letterpaper]{article}  % min is 11pt

\usepackage[sort&compress,numbers]{natbib}
\usepackage[margin=1in,letterpaper]{geometry}  % min is .5 in
\usepackage[shortlabels]{enumitem}
\usepackage{graphicx}
\usepackage{caption}
\usepackage{footmisc}
\usepackage[hidelinks]{hyperref}
\usepackage{fancyhdr}
\usepackage{times}
\rhead{\emph{CZI: Essential Open Source Software for Science}}
\lhead{\emph{Matplotlib: Scientific Visualization}}
%\linespread{1.5}

\setlist[itemize]{noitemsep, topsep=0pt}
\setlist[enumerate]{noitemsep}

\pagestyle{fancy}

\begin{document}
\title{Matplotlib - Foundation of Scientific Visualization in Python}
\author{}
\maketitle

\section{Goals}

Matplotlib\cite{Hunter:2007} is the fundamental
data visualization library for the scientific Python Ecosystem, used
in conjunction with other foundational tools like NumPy and
SciPy \cite{Jones2001,2020SciPy-NMeth} by over a million\footnote{Estimated from \texttt{pypi}
and \texttt{conda} download numbers and unique monthly visitors to the documentation website} users.
Matplotlib is used across a wide spectrum of fields, including bio-medical imaging,
microscopy, and genomics \cite{Carpenter2006,Wolf2018,10.7717/peerj.453,
  Segata2011,10.1371/journal.pgen.1000695,HASHIMSHONY2012666,
  10.1093/bioinformatics/bts480,Carlile2014,Laganowsky2014,Jiangaac9462,
  10.3389/fninf.2014.00014}, and we expect this user base to grow as Python
  continues to be adopted in the life sciences. %%?citation needed that this still happening?
Matplotlib has been actively developed and maintained by a vibrant,
primarily volunteer, community over the last 17 years on mostly volunteer effort; however, given
the scale, scope, and importance of the project, it is critical that we
continue to supplement the volunteer effort with supported developers.

This proposal asks for approximately 2.5 person-years of developer effort to support:

\begin{enumerate}[label=\alph*),noitemsep]
  \item maintaining the library and contributor base;
  \item fostering community, documentation, and tools to grow a rich ecosystem of domain-specific libraries built on Matplotlib;
  \item and beginning the work of evolving the core architecture.
\end{enumerate}

\subsection{Fostering downstream projects}
\label{sec:downstream}

The most common visualizations in a domain need to be fluid for
the end-practitioners, with the ``obvious'' customization options
exposed. Much of the domain-specific specialization is carried in the
structure, semantics and assumptions of the data, and in the standard
visualizations of the domain. These specializations can vary widely,
in contradictory ways, between domains. Because no high-level API can
simultaneously satisfy all of the visualization needs, there will
always be a need for domain-specific visualization libraries.

Through maintaining the code base and developing the architecture for
the next version, we will generate API guidelines so that downstream
libraries inter-operate well and ``feel'' similar. By building out
community, documentation, and tools, we will provide downstream
library developers resources for efficiently building their tools with
Matplotlib.

We will continue to engage with libraries in the life sciences that
are using Matplotlib for their visualization to ensure that we are
addressing their problems. We will collaborate on prototyping an
end-to-end solution, including data, plot representations, and a
high-level user API.  In particular we plan to engage with
\texttt{scikit-learn}\footnote{A Essential Open Source Software for
Science Round 1 grantee}, \texttt{CellProfiler}\footnote{Currently
funded by CZI\label{f:czi}}, \texttt{scanpy}\footref{f:czi},
\texttt{starfish}\footref{f:czi}, \texttt{nipy}, and
\texttt{scikit-image}\footref{f:czi}
\cite{10.7717/peerj.453,Carpenter2006, Wolf2018}.  Our goal is to
domain specific libraries like them easier to write.

\subsection{Maintenance and Growth of Library and Community}

Matplotlib is a community-driven project, but we have grown to the
point where we need supported developers with the time to organize,
plan, and make decisions.  To maintain Matplotlib's health and grow
the library we need to:

%% these can probably be shortened further, to just stuff like Triage and on board
%%since explainations are in text
\begin{itemize}[noitemsep]
\item promptly triage and review newly opened Issues and PRs;
\item fix critical bugs and regressions;
\item implement new features to adapt to the growing needs of science;
\item maintain backward compatibility and extensively document intentional changes;
\item annd on-board new contributors to sustain and diversify developer team;
\end{itemize}

Supported developers has allowed us to begin to address Issues an PRs
at a faster rate than they are being submitted and reduce our backlog.
From early March when Elliott started we have been reducing the number
of open PRs by an average of just over 4 per week while increasing the
total through put from 40-50 to XYZ.  This work has included finishing
long stalled documentation and maintenance work that likely would not
have happened without the current support.

The supported developers will continue to complement and facilitate,
not replace, crucial volunteer work.  The goal is to grow and
sustaining a diverse community of volunteer and paid expert
contributors.


\section{Expected outcomes, success evaluation and metrics}
\subsection{Maintenance}

Quantitatively evaluating maintenance work can be tricky---some Issues
or PRs take minutes to review while others can take days to months of
effort---but we believe that there is value at looking at the number
of open Issues and PRs.  We will reduce this number by closing Issues
and PRs faster than they are opened until a reasonable equilibrium is
reached. With the introduction of paid developers, NumPy has had
success in reversing the ever increasing trends in the number of open
Issues and Pull
Requests\footnote{https://github.com/seberg/numpy\_talk\_plots/blob/master/plots\_used\_in\_talk/issues\_prs\_delta.pdf}. Once
the backlog is handled, we aim to have all new Issues and Pull
Requests labeled within 7 days of being opened.

Improved contributor onboarding should yield an increase in the
percentage of new contributor PRs that get resolved, a decrease in the
amount of time it takes for resolution, and an increase the number of
PRs submitted by non-core developers.


\section{Work Plan}

We request funding for the following:

\begin{itemize}[noitemsep]

\item Fund Thomas Caswell's position at 40\%.  Caswell is the Project Leader of Matplotlib and an Associate
  Computational Scientist at Brookhaven National Laboratory.  His
  long-term experience, API design expertise, and project leadership
  are critical to the success of the work in this proposal.  He will work
  on all aspects of the proposal.
\item Hannah Aizenman's position for 12 months.  Aizenman has
  been a core developer of Matplotlib for four years.
\item Summer support for Michael Grossberg, Aizenman's PhD advisor.
\item 12 months of Elliott Sales de Andrade as a Research Software
  Engineer to support all aspects of the proposal but focusing on
  maintenance, prototyping, and engaging down-stream libraries.  de Andrade is a
  a long time Matplotlib contributor and is currently employed
\item Travel to key conferences (such as SciPy or PyCon) and for in-person meetings
\item Hardware and services to support the
\end{itemize}

We want to use this dedicated effort to leverage and empower the
Matplotlib developer community.  In terms of direct work, equal
amounts of time will be spent mentoring and reviewing code from
community members and directly implementing features or fixing bugs.
All of the design work will be done in public with input from the
community. Part of this work is to develop the project Road Map.



\section{Existing Support}
CZI round 1



\clearpage
\bibliographystyle{ieeetr} % or named ?
\bibliography{biblo}

\end{document}
